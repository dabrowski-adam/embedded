This section describes how the ARM7 port of EFSL works. This documentation was written
by Martin Thomas, as is the port to the ARM7 and the examples included with EFSL.
The examples are pretty large, so we will not print them here, they have their own
subdirectory in the examples sections and should be quite understandable.

Please note that the LPC2000 interface is Copyright (c) by Martin Thomas, Kaiserslautern, Germany.

\subsubsection{License}
The LPC2000-interface inherits the 
LGPL-license from the main efsl-source.


\subsubsection{General information about the ARM7 port}
	\begin{itemize}
\item A GNU arm-elf toolchain/cross-compiler
must be available to build the examples
and the library.

\item The interface supports SD-Cards
connected to the LPC2000 SPI ("SPI0") 
and SSP ("SPI1") interface. The SPI-
interface is hardcoded by the value 
defined in config.h and can not be 
changed during runtime.

\item Only some parts of the LPC2000-family
have the fast SSP-Interface (i.e. LPC213x,
LPC214x) others only provide the slower 
SPI interface.

\item \code{\#define BYTE\_ALIGNMENT} is disabled
for the LPC-interace in config.h 
It didn't work with alignment enabled.

\item Buffers can be configured as
described in the efsl-manual.

\item To build the library libefsl.a for 
LPCs use the makefile Makefile-LPC2000 
in the efsl-root-directory.

\item The hardware connection is similar 
to the connection described
in the efsl-manual for Atmel AVRs.
(I have mounted the pullups in my
test-system but initial test have
been done successfully without the
pull-up resistors)

\item So far the interface has only been
tested with a LPC2138-controller
and the SPI(0) and SSP-Interface.
Other LPC2000-ARM7 controllers 
should be supported too. Verify
that the register-adresses in 
inc/interfaces/LPC2000\_regs.h
match the used controller and the
maximum speed defined in lpc2000\_spi.c
is available for the part.
Verify the pin-connections in lpc2000\_spi.c.
	\end{itemize}

\subsubsection{Example code}

There are two LPC demo-applications
provided in the examples-directory:

\begin{enumerate}
\item Example lpc2138\_ssp\_standalone:

This application gets build with
the efsl source-code. The library
libefsl.a is not needed.

The application demonstrates:
- Init of the efsl LPC2000 debug-system
- Init of the efsl-library
- List contents of the SD-Card's
  root directory
- Open a file and type the content
- Append a line of text to a file
- "cat/type" the file's content again

The application prints status- and
debug-messages to the LPC UART1
(115200,8,N,1,no FC)

\item Example lpc2138\_ssp\_withlib:

Same functions as the standalone
applications but the libefsl.a must
be available for linking.
\end{enumerate}

\subsubsection{Test Hard- and Software}
\begin{itemize}
\item Keil (keil.com) MCB2130 board
\item SD/MMC-connector from Buerklin.de
\item SanDisk "standard" SD-Card 256MB
  (not the "highspeed"-version)

\item WinARM (arm-elf GNU toolchain)
\item Bray++ Terminal
\end{itemize}

\subsubsection{Credits}
\begin{itemize}
\item The LPC2000 debug-printf routine
is based on code from Holger Klabunde
and has been modified to support
a "fdevopen"-function. The original
code has no copyright-notice and
is derived from an free code by
Volker Oth.

\item Register-defintions based on
a header-file from Keil GmbH (supplied
with the DKARM trial-version)

\end{itemize}
