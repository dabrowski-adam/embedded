\documentclass[a4paper, portrait,11pt]{article}
\usepackage[verbose,a4paper,tmargin=3cm,bmargin=3cm,lmargin=2.5cm,rmargin=2.5cm]{geometry}
\usepackage[utf8]{inputenc}
\usepackage{polski}
\usepackage{amsmath}
\usepackage{amsfonts}
\usepackage{amssymb}
\usepackage{lastpage}
\usepackage{indentfirst}
\usepackage{verbatim}
\usepackage{graphicx}
\usepackage{fancyhdr}
\usepackage{listings}
\usepackage{url}
\frenchspacing
\pagestyle{fancyplain}
\fancyhf{}
\renewcommand{\headrulewidth}{0pt}
\renewcommand{\footrulewidth}{0.4pt}
\newcommand{\degree}{\ensuremath{^{\circ}}} 
\fancyfoot[L]{Systemy wbudowane: Paweł Tarasiuk}
\fancyfoot[R]{\thepage\ / \pageref{LastPage}}

\begin{document}

\begin{titlepage}
\begin{center}
\begin{tabular}{c}
\begin{tabular}{|r|}
\hline \\
\large{\underline{151021~~~~~~~~~~~~~~~~~~~~~~~~~~~~~~~~~~~~~~~~~~~} }\vspace{-0.15cm}\\
\emph{\scriptsize numer indeksu} \\
\large {\underline{Paweł Tarasiuk~~~~~~~~~~~~~~~~~~~~~~~~~~~~~~~~} }\vspace{-0.15cm}\\
\emph{\scriptsize imię\ i\ nazwisko} \\\\ \hline
\end{tabular} 
\end{tabular}
~\\~\\~\\ 
\end{center}
\begin{tabular}{ll}

\LARGE{\textbf{Kierunek}}& \LARGE{Informatyka}\\
\LARGE{\textbf{Specjalność}}& \LARGE{Inżynieria oprogramowania i analiza danych}\\
\LARGE{\textbf{Rok akademicki}}& \LARGE{2011/12} \\
\LARGE{\textbf{Semestr}}& \LARGE{7} \\\\\\\\\\
\end{tabular}
\begin{center}
\Huge{Systemy wbudowane}\\\ \\
\textbf{\Huge{Gra "refleks"}}\\
\LARGE{przy wykorzystaniu płytki Eduboard LPC2148}
\end{center}
\end{titlepage}

\section{Opis projektu}
Przygotowany projekt stanowi grę zręcznościową, która wymaga od gracza szybkich
reakcji joystickiem na bodziec, jakim jest ukazywanie się znanych symboli na
matrycy LED. Program wykonywany jest na płytce Eduboard LPC2148 - wystarczy
że jest ona podłączona do zasilania i wgrany jest przygotowany w ramach projektu
program. Jeżeli płytka jest podłączona do partu COM, możliwe jest odczytywanie
dodatkowych informacji logujących za pomocą przygotowanego w ramach projektu
programu działającego po stronie komputera (przesyłanie komunikatów odbywa
się za pomocą przygotowanego w ramach projektu prostego protokołu z sumami
kontrolnymi).\\

Program na płytkę został napisany w języku C i skompilowany za pomocą GNU ARM
z~kompilatorem z linii 3.4. Terminal do odczytywania komunikatów także został
napisany w C, zaś skompilowany został za pomocą kompilatora GNU pod platformą Cygwin.\\

Poniższa tabela opisuje zakres projektu:
\begin{center}\begin{tabular}{|l|l|}
\hline
\textbf{Zakres funkcjonalności} & \textbf{Zastosowanie} \\\hline\hline
Joystick i przycisk P0.14 & Odbieranie ruchów gracza i informacji o gotowości do gry \\\hline
Matryca LED & Wyświetlanie symboli, na które ma reagować gracz \\\hline
Wyświetlacz LCD & Wyświetlanie komunikatów, w tym: wyniku końcowego \\\hline
Dioda RGB & Sygnalizowanie liczby żyć wynikającej z pomyłek gracza \\\hline
Silnik & Wykonywanie ruchu na wiwat, gdy gracz wygra \\\hline
UART & przesyłanie opcjonalnych komunikatów przez port COM \\\hline
\end{tabular}\end{center}

Szczegółowe rozważanie organizacji pracy zespołowej, podziału obowiązków
oraz harmonogramu jest niemożliwe z przyczyn formalnych, ze względu
na specyficzne warunki powstawania projektu.

\section{Przebieg gry}
Po uruchomieniu zaprogramowanego urządzenia oraz po zakończeniu gry, ekran
LCD jest podświetlony i wyświetla napis informujący o tym, że wciśnięcie
przycisku $P0.14$ spowoduje rozpoczęcie gry. Gra polega na tym, że
na matrycy LED wyświetlane są przez pewien czas pseudolosowe wzory,
aż w wyniku losowania program wykona instrukcję wyświetlającą
zapytanie do użytkownika - to znaczy, symbol jednej z czterech strzałek
na płaszczyźnie, albo kojarzący się z~lotką strzały symbol wektora
wskazującego "za diagram". Zapytanie jest wyświetlane do czasu, aż
gracz wykona odpowiadający mu ruch joystickiem - jednocześnie
zapamiętywane są stany regularnie inkrementowanego przez timer
licznika w chwili wyświetlenia zapytania oraz uzyskania odpowiedzi.
Po 10 poprawnych odpowiedziach, wynikiem ostatecznym jest średni
czas reakcji gracza na zapytania - wynik zostaje wypisany na wyświetlaczu LCD,
który po zakończonej grze znów jest podświetlony. Zakończenie gry sukcesem
jest "nagradzane" hałaśliwym ruchem silnika. Podczas gry raz może dojść
do sytuacji, że gracz wykona ruch niepasujący do zapytania (albo
ruch w momencie, kiedy żadnego zapytania nie ma, a matryca wyświetla
obraz pseudolosowy). Drugi błąd oznacza zakończenie gry i porażkę.
Liczbę "żyć" gracza (2, 1 lub 0) sygnalizuje dodatkowo napis
na niepodświetlonym ekranie LCD (aby gracz mógł się skoncentrować
na matryce LED, nie zaś na podświetleniu wyświetlacza) oraz kolor
diody RGB: zielony na początku gry, żółty po jednym błędzie,
oraz czerwony w przypadku porażki.

Jeżeli przez RS232 płytka zostanie podłączona do portu COM komputera,
na którym uruchomiony zostanie przygotowany w ramach projektu
terminal, możliwe będzie wygodne odczytywanie dodatkowych
informacji o czasach reakcji na poszczególne zapytania,
liczbie żyć oraz wyniku końcowym. Aby od razu przekonać się,
że połączenie działa, na początku każdej rozgrywki wyświetlany
jest ponadto komunikat powitalny.

\section{Opis algorytmu}
Program po uruchomieniu czeka na wciśnięcie przycisku P0.14 (w standardowy
sposób badając jego stan przez GPIO). Automatycznie przy starcie
programu tworzona jest także procedura odpowiadająca za zmiany
w danych służących do generowania pseudolosowych ciągów bitów.

\subsection{Liczby pseudolosowe}
Zastosowana metoda jest podobna do Blum-Blum-Shub, problemem
jest jednak brak prawdziwie losowego ziarna. Dlatego w 16-bitowej
zmiennej "radnom" przechowywana jest pewna liczba, która wykorzystywana
jest w następujący sposób:

\begin{itemize}
\item Przy generowaniu losowego bitu, jest ona podnoszona do kwadratu
modulo $64829$ (16-bitowy iloczyn dwóch możliwie dużych liczb pierwszych).
Większe liczby losowe generowane są jako ciągi tworzonych w ten sposób
losowych bitów.
\item Jeden z tworzonych procesów odpowiada za psucie zmiennej random co
pewien czas w~możliwie chaotyczny sposób - aby ciągi bitów nie powtarzały
się zbyt często (operacje wykonywane są na zbyt małych liczbach, aby
metoda Blum-Blum-Shub działała dobrze) i~całość była mniej przewidywalna.
\end{itemize}

\subsection{GPIO}
Obsługa przycisku P0.14 do startowania gry jest bardzo prosta
- po ustawieniu odpowiedniego bitu $IODIR$ na 0 (odczyt),
badana jest wartość bitu $IOPIN$ o odpowiedniej pozycji
- gdy wynosi ona $0$, to znaczy, że przycisk jest wciśnięty.

W przypadku joysticka zastosowany został ciekawszy sposób
badania wielu bitów naraz (konkretnie: P0.16 do P0.20).
W każdej iteracji zapamiętywana jest poprzednia wartość
$IOPIN$ w 32-bitowej zmiennej $oldIOPIN$, po czym zapisywana
w zmiennej jest wartość wyrażenia \url{(oldIOPIN & ~IOPIN) & 0x001f0000}.
Jeżeli nie jest ona zerem, to stan joysticka zmienił się w~porównaniu
z poprzednią iteracją (pewne bity zmieniły stan z 1 na 0) -
badając poszczególne bity wyznaczonej w ten sposób zmiennej łatwo
ustalić, który ruch został wykonany.

\subsection{Timer}
W projekcie wykorzystany został TIMER1, który ustawiany jest przy
wykorzystaniu mechanizmu przerwań. Zarządzaniem przerwaniami
zajmuje się Vectored Input Controller (VIC), który posiada
32 sloty przypisane urządzeniom mogącym generować przerwania
(zaliczają się do tego timery).

Timer jest ustawiony tak, aby generował przerwania typu
IRQ. Funkcja ustawiająca timer jest wywoływana tak, aby
co $2\,\mbox{ms}$ wykonywana była procedura odpowiedzialna
za aktualizację stanu matrycy LED oraz podbicie specjalnego
licznika, którego wartości są porównywane w celu ustalenia
ile czasu upływa od zapytania do reakcji gracza.

\subsection{Matryca LED}
Do obsługi wyświetlacza diodowego wykorzystywany jest
szeregowy interfejs urządzeń peryferyjnych (SPI).

Za pomocą linii CS (Chip Select) wybierany jest odpowiedni
układ (w tym wypadku - kolumna 8 diod), po rozpoznaniu którego
przesyłana jest porcja danych (dla wyświetlacza diodowego - bajt
którego bity opisują stan poszczególnych diod).

Wywoływana raz na $2\,\mbox{ms}$ procedura ustawiająca stan diod
za każdym razem ustawia jedną kolumnę, zatem odświeżenie całego
wyświetlacza trwa $16\,\mbox{ms}$.

\subsection{Wyświetlacz LCD}
Wyświetlacz wierszowy LCD pozwala wyświetlić dwa wiersze po
15 znaków tekstu. Do kontroli wyświetlacza wykorzystany
został mechanizm GPIO. Sterownik wyświetlacza LCD
posiada następujące rejestry:

\begin{itemize}
\item ControlWR – sterowanie (P0.22)
\item DataWR – zapis danych do sterownika (P1.16 – P1.23)
\item ControlRD – sterowanie i status wyświetlacza (P1.24)
\item DataRD – odczyt danych ze sterownika (P1.25)
\end{itemize}

Pin P0.30 pozwala ponadto sterować podświetleniem wyświetlacza.\\

W ramach projektu do wypisywania na wyświetlaczu całych napisów
przygotowana została jedna funkcja, która w argumentach
przyjmuje napis oraz informację o tym, czy wyświetlacz
ma pozostać podświetlony (zawsze jest podświetlany podczas
pojawiania się informacji, aby zasygnalizować że stało
się coś nowego). Szczególnej obsługi wymagał znak
przejścia do nowej linii, który powoduje że dalsze pisanie
odbywa się w drugim wierszu wyświetlacza.

\subsection{Silnik}
Silnikiem zamontowanym na płytce można sterować za pomocą
modulacji szerokości impulsów (PWM).

Inicjalizacja silnika polega na ustawieniu funkcji PWM4
i PWM6 dla pinów P0.8 i P0.9, a~następnie wyłączenie
silnika poprzez ustawienie stanu niskiego na pinie P0.10.
Następnie możliwe jest ustalenie następujących ustawień:

\begin{itemize}
\item Prescale Counter ustawiony na 0 będzie oznaczał inkrementację PWMTC przy każdym takcie zegara
\item Match Control Register pozwala ustawić, aby PWMTC był resetowany, gdy PWMMR0 = PWMTC
\item W rejestrze PWMMR0 ustawiana jest szerokość impulsu: 0x1000
\item Początkowe wartości PWM\_MR4 i PWM\_MR6 to 0, aby stosunek sygnału wysokiego do niskiego wynosił 0 i silnik się nie obracał wcale.
\item Za pomocą Latch Enable Register należy zatwierdzić zmiany wartości PWM\_MR4 i PWM\_MR6.
\item Ustawiony za pomocą PWM\_PCR tryb "single edge" będzie oznaczał że stan wysoki będzie się utrzymywał aż do resetu licznika.
\item Poprzez rejestr PWM\_TCR dokonywane jest uruchomienie licznika PWM Timer Counter i włączenie trybu PWM.
\end{itemize}\ \\

Później wystarczy operować szybkością obrotów silniczka za pomocą rejestrów PWM\_MR4 i PWM\_MR6 i zatwierdzać zmiany
za pomocą PWM\_LER.

\subsection{Dioda RGB}
Dioda RGB, podobnie jak silnik - obsługiwana jest przez PWM. Mechanizm postępowania oraz sam
kod odpowiadający za obsługę PWM dla diody jest łudząco podobny do tego, który jest związany
z silnikiem. Ustawienia pinów dla poszczególnych rejestrów PWM to:
PWM2 dla P0.7 określa poziom światła czerwonego, PWM4 dla P0.7 - niebieskiego, zaś PWM6 dla P0.9
- zielonego.

Należy zauważyć, że związek między diodą a silnikiem jest bardzo duży, gdyż reagują na PWM
w ten sam sposób - dlatego obroty silniczka w jedną lub drugą stronę muszą być związane
ze świeceniem diody odpowiednio na niebiesko i na zielono, zaś gwałtowne zmiany kolorów
diody powinny być wykonywane przy wyłączonym silniku (tak jest to zrobione w projekcie).
Niebieskie zabarwienie diody podczas ruchu silnika po wygranej grze nie kłóci się
jednak w żaden sposób z instrukcją rozgrywki i znaczeniami kolorów diody - można
przyjąć, że to element "fanfar" dla zwycięzcy.

\subsection{Komunikacja z terminalem}
Do przesyłania danych dla terminala przez port COM wykorzystane są funkcje zawarte w~przygotowanym
przez Embedded Artists frameworku, takie jak uproszczony w stosunku do standardu C, lecz
mający podobną składnię printf. Interfejs RS232 zawarty na płytce połączony jest z UART\#0.\\

W ramach inicjalizacji UART odpowiednie piny GPIO przestawiane są na funkcje UART: RxD0 i TxD0
odpowiedzialne odpowiednio za odbieranie i przesyłanie danych:\\
\url{PINSEL0 = (PINSEL0 & 0xfffffff0) | 0x00000005;}\\

Chcąc korzystać z UART należy wyznaczyć współczynnik będący pomnożonym przez 16 ilorazem
częstotliwości zegara procesora (60 MHz) przez szybkośś transmisji (38400).\\
\url{#define} \url{UART_DLL_VALUE (unsigned short)((PCLK / (CONSOL_BITRATE * 16.0)) + 0.5)}\\

Następnie wartość umieszczana jest w rejestrach DLL i DLM; wcześniej ustawiamy bit DLAB w rejestrze LCR:\\
\url{UART_LCR = 0x80;}\\
\url{UART_DLL = (unsigned char)(UART_DLL_VALUE & 0x00ff);}\\
\url{UART_DLM = (unsigned char)(UART_DLL_VALUE>>8);}\\

Na koniec ustalany jest format danych opisywany przez bity w rejestrze LCR. Ustawienie dla 8 bitów danych
i 1 bitu stopu, bez parzystości to:\\
\url{UART_LCR = 0x03;}\\

Znaki mogą być później wysyłane przy wykorzystaniu rejestru UART\_THR.

\section{Analiza skutków awarii}
Awarie o krytycznych skutkach:
\begin{itemize}
\item Joystick i przycisk, których stany odczytywane są przez GPIO
- ich awaria uniemożliwiłaby komunikację z graczem.
\item Awaria znacznej części diod z matrycy LED - gdyby symbole
przestały być czytelne, łatwo można by je było pomylić z wzorami
losowymi, nawet gdyby pozostały rozróżnialne między sobą.
\item Timer - jego awaria spowodowałaby brak odświeżania wyświetlacza
diodowego oraz brak pomiarów czasu.
\item Jednoczesna awaria wyświetlacza LCD i transmisji przez
port COM - uniemożliwiłaby odczytanie wyników czasowych, przez
co gracz mógłby uznać grę za bezsensowną.
\end{itemize}\ \\

Awarie o poważnych skutkach:
\begin{itemize}
\item Awaria nieznacznej części diod z matrycy LED - komfort
gry byłby znacznie zaburzony, gdyby symbole były wyświetlane
niedokładnie - w szczególności, gracz miałby prawo myśleć,
że to losowy wzór ułożył się w kształt podobny do symbolu
zapytania i przez to go z~początku zignorować, pogarszając
swój wynik.
\item Wyświetlacz LCD - gracz musiałby wiedzieć o konieczności
rozpoczęcia gry przyciskiem P0.14. Ponadto odczytanie wyników
rozgrywki byłoby możliwe tylko przez port COM (co jest mniej
wygodne, bo wymaga komputera).
\end{itemize}\ \\

Awarie o niegroźnych skutkach:
\begin{itemize}
\item Dioda RGB - stanowi tylko dekorację a sygnalizowane przez
nią informacje można odczytać także w inny sposób.
\item Silnik - brak kręcenia się na wiwat - po prostu bez
dźwięku silnika po wygranej grze zaburzony zostałby klimat
gry.
\end{itemize}\ \\

\section*{Bibliografia}
\begin{itemize}
\item LPC214x User Manual - \url{http://ics.nxp.com/support/documents/microcontrollers/pdf/user.manual.lpc2141.lpc2142.lpc2144.lpc2146.lpc2148.pdf}.
\item LPC2148 Education Board User's Guide - \url{http://www.zsk.p.lodz.pl/~morawski/SCR&ES/Guides&Schematics/LPC2148_Education_Board_Users_Guide-Version_1.2_Rev_B.pdf}.
\item \url{http://downloads.sourceforge.net/project/lpc21isp/lpc21isp/1.83/lpc21isp_183.tar.gz} - lektura kodu źródłowego była pomocna przy tworzeniu własnego terminala.
\end{itemize}
\end{document}
